\documentclass{article}
\usepackage[english]{babel}	

\usepackage{float}
\usepackage[singlelinecheck=false]{caption}
\usepackage{geometry}

% syntax highlighting
\usepackage{listings}
\usepackage[usenames,dvipsnames]{xcolor} % Required for custom colors
\lstloadlanguages{Ruby}
\lstset{language=Ruby,
	basicstyle=\ttfamily\color{black},
	keywordstyle=\ttfamily\color{blue},
	commentstyle = \ttfamily\color{gray},
	stringstyle=\color{MidnightBlue},
	numbers=left, 
	firstnumber=1, 
	numberstyle=\color{Blue}, 
	stepnumber=1 ,
	tabsize=2, 
}

% Custom sectioning
\usepackage{sectsty}
\allsectionsfont{ \normalfont\scshape}

% horizontal rule
\newcommand{\horrule}[1]{\rule{\linewidth}{#1}}

% table fix
\restylefloat{table}

\title{
	\normalfont \normalsize \textsc{CSCI 123: Course Name} \\ [25pt]
	\horrule{0.5pt} \\[0.4cm]
	\huge Title \\
	\horrule{2pt} \\[0.5cm]
}

\date{\today}

\author{
	First Last
	\vspace{.5in}
}

\setlength\parindent{0pt}
\begin{document}
\maketitle
\clearpage

\section{Section}
Text with fancy \LaTeXe

\subsection{Subsection}
Equation: \\
\begin{equation}
M[i,j]=\textrm{Max} \{M[i-1,j-k]+k(p_i-f_k-T[i-1,j-k]     k=0...j\}
\end{equation} \\

\begin{enumerate}
\item This
\item is
\item a
\item[] list % no index
\end{enumerate}

\begin{table}[H]
\begin{tabular}{|l|l|l|l|l|l|}
\hline
\textbf{$M[i,j]$}      & \textbf{0} & \textbf{1}    & \textbf{2}     & \textbf{\ldots} & \textbf{x}     \\ \hline
\textbf{1}      & 0          & $(p_1 - f_1)$ & $2(p_1 - f_2)$ & \ldots          & $x(p_1 - f_x)$ \\ \hline
\textbf{2}      & 0          &               &                &                 &                \\ \hline
\textbf{\ldots} &            &               &                &                 &                \\ \hline
\textbf{n}      & 0          &               &                &                 &                \\ \hline
\end{tabular}
\caption{This is a caption}
\end{table}

\begin{lstlisting}[language=Ruby, caption=Ruby snippet]
#this is a comment

class foo
	def bar
		puts "foobar"
	end
end

\end{lstlisting}

\end{document}